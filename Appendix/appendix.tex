\documentclass[../main.tex]{subfiles}

\begin{document}


\section{Appendix}

\subsection{The cause of the ``strategic effect''}

The main difference between our equilibrium and that of \cite{schnell2017physician} is the presence of a ``strategic effect'', wherein doctor $j$ takes into account the behavior of other doctors in the selection of her own $\bar{\kappa_j}$. Of the different modifications we made to her framework, we'll argue it's the absence of \textit{additive separability} across patients in physician utility $U_j(\cdot)$ which accounts for this.

In our context of unbounded maximization by doctor $j$, the absence of additive separability implies she can't consider each patient individually when it comes to whether she's willing to allow (or induce) their visit, as such a decision is no longer independent of other patients' visits; the marginal utility of an additional patient is dependent on the aggregate of clients up to that point, both in terms of the visit itself as well as in the number of sick leave certificates granted up to that point.

Let's illustrate this point. Consider for a moment a finite number of patients $1, ... , k$, where each patient is inputed directly as an argument in our doctor $j$’s $U_j(\cdot)$, like so: $U_j(1, ..., k)$. If $U_j(\cdot)$ has the property of additive separability, this means it may be reformulated like so:
\[
U_j(1, ...,k) = v_{j1}(1) + ... + v_{jk}(k) = \sum_{i = 1}^{k} v_{ji}(i)
\]
Unconstrained optimization in this context implies she’s willing to see any patient whose $v_{ji}(i)$ is non-negative, such that her optimal level of utility is:
\[
U_j^*(1, ...,k) =  \sum_{i : \; v_{ji}(i) \geq 0} v_{ji}(i)
\]
In the context of our doctor-patient model, where physician utility is increasing in $\kappa_i$, selection is achieved by the doctor by choosing a $\kappa_j^*$ which excludes all patients $i$ whose $\kappa_i < \kappa_j^*$. Supposing our patients are well-ordered in $\kappa_i$, the choice of such a $\kappa_j^*$ would be one where the marginal consumer $i$ affords a non-negative $v_{ji}(i)$, and the inframarginal consumer $i + 1$ fulfills $v_{j,i+1}(i+1) < 0$. Ignoring for a moment that patients themselves have a \textit{choice} of visiting -- depending upon a second dimension $\gamma$ --, we would then have:
\[
U_j^*(1, ...,k) =  \sum_{i : \; \kappa_i \geq \kappa_j^*} v_{ji}(i)
\]

If instead of a discrete set we consider a mass of consumers $\mathcal{I}$ characterized by their level of $\kappa_i$, and we make the simplifying assumption that $v_{ji}$ takes on the same form $v_{j}$ for every $i \in \mathcal{I}$, our $U_j(\cdot)$ could be expressed as:
\[
U_j^*(\mathcal{I}) =\int_{\kappa_j^*}^{\infty} v_{j}(k) \, dF(k)
\]

$U_j^*(\mathcal{I})$ represents the optimal value of $U_j(\mathcal{I})$, where doctor $j$ only sees patients who provide her with non-negative marginal utility, i.e. such that $\kappa_i \geq \kappa_j^*$. We can find this optimal value $\kappa_j^*$ by looking at \textit{threshold} equilibria, where doctor utility is also dependent on the threshold $\bar{\kappa_j}$ over which she’s willing to see patients:
\[
U_j(\mathcal{I}, \bar{\kappa_j}) =\int_{\bar{\kappa_j}}^{\infty} v_{j}(k) \, dF(k)
\]

The optimal value of threshold $\bar{\kappa_j}$ is $\kappa_j^*$. Assuming $U_j(\cdot) $ is twice differentiable and concave in $\bar{\kappa_j}$, such a solution may be arrived at through the FCO:
\[
\frac{\partial U_j(\mathcal{I})}{\partial \bar{\kappa_j}} \equiv - v_{j}(\bar{\kappa_j})f(\bar{\kappa_j}) = 0
\]
Solving for this would yield $\bar{\kappa_j} = \kappa_j^*$.

\cite{schnell2017physician} is an example of just such a treatment, which specifies the physician's additive utility by patient in the following manner:
\[
v_{ji}(\kappa_i) \equiv R_j + \beta_j h(\kappa_i)
\]
where $R_j$ is a parameter standing for revenue by visit, and $\beta_j h(\kappa_i)$ represents the physician's ``altruistic'' utility over the health impact of a prescription drug to a patient with ``pain level'' $\kappa_i$.

The optimal threshold $\kappa_j^*$ is then obtained out of the maximization\footnote{Once again, ignoring patient choice and $\gamma$,. We're presenting a \textit{simplified} version for illustrative purposes.}:
\[
\max_{\bar{\kappa_j}}\int_{\bar{\kappa_j}}^{\infty}  R_j + \beta_j h(k) \, dF(k)
\]

Which gives out the following FOC:
\[
R_j = - \beta_j h^{\prime}(\bar{\kappa_j})
\]
which, as is immediately apparent, doesn't depend upon the behavior of other doctors ---more specifically, \textit{their} choice of $\bar{\kappa_j}$. It merely establishes that, at the threshold, marginal benefit by patient --revenue $R_j$-- must equal marginal ``cost'' --altruistic ``cost'' $\beta_j h(\bar{\kappa_j})$--. A way to interpret this is that doctors will be willing to see any and all patients which render them positive marginal utility, unconcerned with their total market share, and thus, what other physicians may do to take away clientele.

Such a treatment is rendered inviable by our choice of utility function. Schnell's parameter of revenue would in our model imply the linearity of our revenue \textit{function} $R_j(\cdot)$ and our $P(\cdot)$ function over \textit{aggregate} licenses granted. Strict convexity of $P(\cdot)$ would forestall its formulation as Schnell-like $\beta_j h^{\prime}(\bar{\kappa_j})$ terms for each patient, because the impact on doctor $j$'s utility in granting patient $i$ a license would no longer independent from the granting of licenses of other patients. Likewise, strict concavity of $R_j(\cdot)$ belies a simple ``r'' parameter such that $R_j'(Q_j) = r Q_j$.

More formally, when $U_j(1,...,k)$ \textit{isn't} additively separable across patients, the value $\kappa_j^*$ such that if $\kappa_i \geq \kappa_j^*$ patient $i$ provides positive marginal utility, isn’t independent of current clientele, because what before was a properly defined object, marginal utility by patient $i$, $v_j(\kappa_i)$, can no longer be so identified. The marginal utility $i$ provides to $j$ as the $k$th client (assuming some order over clients) is not necessarily the same he’d provide as the $k+1$th client, and so, as the $k$th client he could provide $0$ utility, impliying $\kappa_i = \kappa_j^*$, whereas as the $k + 1$th he could be inframarginal, such that $\kappa_i < \kappa_j^*$.

$\kappa_j^*$ is not longer independent of clientele mass $\mathcal{I}$ as before, but a function of it, $\kappa_j^*(\mathcal{I})$. More specifically, in the models we consider it will depend on the \textit{cardinality} of current clientele, $|\mathcal{I}|$, such that our physician's choice of marginal consumer will depend on her aggregate level of patient demand, where before it didn't. This effect is introduced through the \textit{strict non-linearity} of our physician utility function $U_j(\cdot)$, either through the strict concavity of $R_j(\cdot)$ over expected patient demand $Q_j$, or the strict convexity of $P_j(\cdot)$ over total expected sick leaves granted.

When either of those is the case, $aggregate$ levels enter into the equation and form part of the optimality condition. Our general FOC reflects this:
\begin{equation}
 R_j^{\prime}(Q_j)\frac{\partial Q_j}{\partial\bar{\kappa_j}} = P_j^{\prime}(X_j)\frac{\partial X_j}{\partial \bar{\kappa_j}} 
 \label{eq:apx_foc}
\end{equation}
where either or both of $R_j'(\cdot)$ and $P_j'(\cdot)$ are a non-constant function over aggregates.

When such aggregates, either $Q_j$ or $X_j$, come into play, physicians come to consider their \textit{market share}, which doesn't depend exclusively on their own choice of $\bar{\kappa_j}$. Each patient's strategy $S_i$ is constructed taking into account the whole of $\{(V_j,\bar{\kappa_j})\}_{i =1}^{J}$.

Imagine equation (\ref{eq:apx_foc}) holds for some doctor $j$, and some doctor $l \neq j$ decides to lower her $\bar{\kappa_l}$, enough that it changes the value of $s_{ij}$ for some mass of clients. The value of $Q_j$ would then change, and therefore that of $ R_j^{\prime}(Q_j)$. If $P_j^{\prime}(X_j)$ didn't vary by the same amount, (\ref{eq:apx_foc}) would no longer be an equality, leading $j$ to modify her choice of $\bar{\kappa_j}$ to make equality hold.

This intelligible line of reasoning links the presence of a ``strategic effect'' to the non-additive separability of $U_j(\cdot)$: doctor $j$ takes into account other physicians' strategy in her own choice of $ \bar{\kappa_j}$ because of the present of \textit{aggregate} amounts of clientele in her optimality conditions, which is so because utility isn't additively separable across clients.

\subsection{\cite{schnell2017physician} with Logit choice}

To prove our point that it is additive separability which accounts for a possible strategic effect, we reformulate \cite{schnell2017physician} in the manner of a McFadden Logit -- with some quirks.

The way in which Schnell devises her physician's utility formula is implicitly Bernoulli-like:
\[
\int_{\kappa} \int_{\gamma} u(\kappa, \gamma) \cdot p(\kappa, \gamma) \, dG(\gamma) dF(\kappa)
\]
where $u(\kappa, \gamma)$ is the utility function of the physician from patients characterised by a given $(\kappa, \gamma)$ tuple, and $p(\kappa, \gamma)$ stands for the proportion of clients atomized in such a tuple the physician expects to have visit her.

$u(\kappa, \gamma)$ we'll leave as Schnell defined it: $\beta_j h(\kappa) + R_j$. $p(\kappa, \gamma)$ lends itself nicely as the $s_{ij}$ we have been using thus far, which implicitly depends on the $(\kappa_i, \gamma_i)$ tuple which characterizes patient $i$:
\[
\int_{\kappa} \int_{\gamma} [\beta_j h(\kappa) + R_j] \cdot s_{ij}(\kappa, \gamma) \, dG(\gamma) dF(\kappa)
\]


In our modeling section, our ``implicit'' search model was a left-censored Logit choice model, which gave null probability of assignment to physicians from which patient $i$ expected non-positive utility. It was defined as:
\[
s_{ij} = \frac{\alpha_{ij}}{\sum_{k = 1}^{J} \alpha_{ik}}, \; \; \text{where } \; \alpha_{ij} = \begin{cases}
e^{\lambda u_{ij}}, \; \; \text{if } \; \; u_{ij} > 0 \\
0 , \; \; \text{if } \; \; u_{ij} = 0
\end{cases}
\]

Schnell's opioid-focused physician model provides an additional simplification: patients which aren't given a drug prescription, i.e. those such that $\kappa_i < \bar{\kappa_j}$, don't garner positive utility from a visit. As such, in the original model as in this reformulation, patients ``below'' $\bar{\kappa_j}$ have a null probability of visit, such that the inferior limit of integration for $\kappa$ is $\bar{\kappa_j}$.

As for $\gamma$, the limit of the inner integral remains the same as in Schnell in the absence of a secondary market, a value of $\gamma_i$ such that $u_{ij} = 0$, i.e. $h(\kappa_i) + \gamma_i - \tau^d - \tau^o = 0$.\footnote{Schnell splits the costs a patient will face into costs of visit ($\tau^d$), cost of purchase ($\tau^o$) and search cost ($\tau^s$)}

The double-integral maximized as physician utility is then as follows:
\[
\max_{\bar{\kappa_j}}\int_{\bar{\kappa_j}}^{\infty} \int_{\tau^d - \tau^o - h(\kappa)}^{\infty} [\beta_j h(\kappa) + R_j] \cdot \frac{e^{\lambda u_{ij}}}{\sum \limits_{k : \, u_{ik} > 0} e^{\lambda u_{ik}}} \, dG(\gamma) dF(\kappa)
\]

The FOC of such an equation is:
\begin{equation}
[\beta_j h(\bar{\kappa_j}) + R_j]\int_{\tau^d - \tau^o - h(\bar{\kappa_j})}^{\infty} \frac{e^{\lambda u_{ij}} \mid \bar{\kappa_j}}{\sum \limits_{k : \, u_{ik} > 0} e^{\lambda u_{ik}} \mid \bar{\kappa_j}} \, dG(\gamma) f(\bar{\kappa_j}) = 0 
\label{eq:schnell_logit}
\end{equation}


Where we use ``$\mid \bar{\kappa_j}$'' to clumsily indicate that we integrate $\gamma$ over patients where $\kappa_i = \bar{\kappa_j}$. As an integral over a strictly positive value and the atom of a density function, respectively, the factors B and C in the equation are non-negative. For this reason, the only way for (\ref{eq:schnell_logit}) to be fulfilled is the following condition:
\begin{equation}
    R_j = - \beta_j h(\bar{\kappa_j})
    \label{eq:schnell_condition}
\end{equation}

This is the same result as in the original model in \ref{schnell2017physician} in the absence of a secondary market, a condition which stipulates that marginal utility in $\bar{\kappa_j}$ must be $0$, i.e. that revenue ($R_j$) must equal "altruistic" loss ($\beta_j h(\bar{\kappa_j})$), which pre-supposes that the marginal patient granted prescription opioids suffers a net loss in utility (the negative externalities outweigh its medical benefit as a palliative).

Condition (\ref{eq:schnell_condition}) means a value  $\kappa_j$ is chosen irrespective of the strategies employed by other doctors, it quite literally doesn't enter into the equation. As we have argued, additive separability over clients means the \textit{total} demand for physician $j$'s services doesn't influence marginal utility by a given patient $i$, and so has no sway in the optimality condition. Doctor $j$ states simply that she won't grant a prescription (or sick leave, as in our case) below $\bar{\kappa_j}$, \textit{come what may}. In order for her to care about \textit{aggregate} values, like her total expected demand, making her care about her market share and thus about the strategies of other doctors, \textit{strict non-linearity} must be introduced into her utility function.


\subsection{On the strategic complementarity of $\bar{\kappa_j}$'s}

We had defined the components $s_{ij}$ of $S_i$  as:
\[s_{ij} \equiv g_i(u_{ij}, u_{i,-j})\]
where $g_i(\cdot)$ is a continuous function weakly increasing in the first argument $u_{ij}$. Our defining $s_{ij}$ this way has two interlinked corollaries:

\begin{corollary}
\[s_{ij} \mid \kappa_i < \bar{\kappa_j} \;\;\;\; \leq     \;\;\;\; s_{ij} \mid \kappa_i \geq \bar{\kappa_j} \;\;\;\; \text{(with strict inequality if $\gamma_i > 0$).}\]
\end{corollary}

\begin{proof}
    For a fixed value of $V_j$ and $\kappa_i$, for all $i$, the value of $U_i(V_j, \bar{\kappa_j})$ is $V_j \kappa_i - \tau_j$ if $\kappa_i < \bar{\kappa_j}$, and $\gamma_i + V_j \kappa_i - \tau_j$ if $\kappa_i \geq \bar{\kappa_j}$, where $\gamma_i \geq 0$. Given that $s_{ij}$ is weakly increasing in $u_{ij}$, and we see that $u_{ij}$ is weakly higher if $\kappa_i \geq \bar{\kappa_j}$, the corollary follows.
\end{proof}


\begin{corollary}
\begin{align*}
    \frac{\Delta s_{ij}}{\Delta\bar{\kappa_{j}}} \leq 0 &&  \frac{\Delta s_{ij}}{\Delta\bar{\kappa_{l}}} \geq 0, \forall l \neq j
\end{align*}   
\end{corollary}

\begin{proof}
From the argumentation in corollary 1 follows that $u_{ij}$ is weakly decreasing in $\bar{\kappa_j}$, $\forall j$. Take some $l \neq j$, then $s_{ij}$ is defined in turn as weakly decreasing in $u_{il}$, which implies it is increasing in $\bar{\kappa_l}$.
\end{proof}

Both corollaries hinge upon our definition of $U_i(\cdot)$ as a step function over $\kappa_i$, such that it is discontinuous at $\kappa_i = \bar{\kappa_j}$, where there's a discrete jump of magnitude  $\gamma_i \geq 0$.

\begin{corollary}
\begin{align*}
    \frac{\partial Q_{j}}{\partial\bar{\kappa_{j}}},\frac{\partial X_{j}}{\partial\bar{\kappa_{j}}} \leq 0 &&  \frac{\partial Q_{j}}{\partial\bar{\kappa_{l}}}, \frac{\partial X_{j}}{\partial\bar{\kappa_{l}}} \geq 0, \forall l \neq j
\end{align*}
\end{corollary}

\begin{proof}
We recall our function of patient demand for physician $j$ is:
    \[
    Q_j(\bar{\kappa_j}, \bar{\kappa}_{-j}) \,=\, \int_{0}^{\infty} \int_{0}^{\infty} s_{ij} (k, \gamma) \,dG(\gamma) \,dF(k) 
    \]
Were physician $j$ to increase her threshold from $\bar{\kappa_j}$ to $\bar{\kappa_j}' = \bar{\kappa_j} + \epsilon, \epsilon > 0$, for the mass of patients whose $\kappa_i \in [\bar{\kappa_j}, \bar{\kappa_j} + \epsilon)$, utility would weakly decrease and thus so would their $s_{ij}$, falling to ${s_{ij}}^{\prime} \leq s_{ij}$ (see first corollary, think of ${s_{ij}}^{\prime}$ as $s_{ij} \mid \kappa_i < \bar{\kappa_j}$ as opposed to $s_{ij} \mid \kappa_i \geq \bar{\kappa_j}$).

The difference that would make for patient demand \textit{caeteris paribus} given the $\bar{\kappa}_{-j}$ strategies of the other physicians (ommitted in our notation) is:\footnote{Besides simplifying $Q_j(\bar{\kappa_j}, \bar{\kappa}_{-j})$ into $Q_j(\bar{\kappa_j})$, we also forego the arguments of $s_{ij} (k, \gamma)$, as we've done many times already, to avoid notation clutter. }
\[
\Delta Q_j = Q_j(\bar{\kappa_j} + \epsilon) - Q_j(\bar{\kappa_j})=\int_{\bar{\kappa_j}}^{\bar{\kappa_j} + \epsilon} \int_{0}^{\infty} \{ {s_{ij}}^{\prime} - s_{ij}\} \,dG(\gamma) \,dF(k)
\]

By the definition of a partial derivative:\footnote{The notation for ${s_{ij}} (\bar{\kappa_j}, \gamma)^{\prime}$ is to be taken to mean the strategy ${s_{ij}}^{\prime} \leq s_{ij}$ of the mass of patients at different levels of $\gamma$ atomized at $\kappa_i = \bar{\kappa_j}$.}
\begin{align*}
    \frac{\partial Q_{j}}{\partial\bar{\kappa_{j}}} & = \lim_{\epsilon \rightarrow 0} \frac{\int_{\bar{\kappa_j}}^{\bar{\kappa_j} + \epsilon} \int_{0}^{\infty} \{ {s_{ij}}^{\prime} - s_{ij}\} \,dG(\gamma) \,dF(k)}{\epsilon} \\
    & = \int_{0}^{\infty} \{ {s_{ij}} (\bar{\kappa_j}, \gamma)^{\prime} - s_{ij} (\bar{\kappa_j}, \gamma)\} \,dG(\gamma) \,f(\bar{\kappa_j}) 
\end{align*}
where the derivative is negative if ${s_{ij}}^{\prime}$ is strictly lower than $s_{ij}$, meaning a shift in leniency would affect the utility and optimal strategy of some positive mass of patients around $\bar{\kappa_j}$; or null, if ${s_{ij}}^{\prime} = s_{ij}$.

In a similar vein, consider some physician $l \neq j$ shifting her threshold from $\bar{\kappa_{l}}$ to $\bar{\kappa_l}' = \bar{\kappa_l} + \epsilon, \epsilon > 0$. The fall in patient strategies in $[\bar{\kappa_j}, \bar{\kappa_j} + \epsilon)$ from $s_{il}$ to ${s_{il}}^{\star} \leq s_{il}$ could provide a windfall for $j$ (and all other physicians), raising $s_{ij}$ to ${s_{ij}}^{\star} \geq s_{ij}$ (see previous corollary). In that case:
\[
\Delta Q_j = \int_{\bar{\kappa_l}}^{\bar{\kappa_l} + \epsilon} \int_{0}^{\infty} \{ {s_{ij}}^{\star} - s_{ij}\} \,dG(\gamma) \,dF(k) 
\]
thus rendering:
\begin{align*}
    \frac{\partial Q_{j}}{\partial\bar{\kappa_{l}}} & = \lim_{\epsilon \rightarrow 0} \frac{\int_{\bar{\kappa_l}}^{\bar{\kappa_l} + \epsilon} \int_{0}^{\infty} \{ {s_{ij}}^{\prime} - s_{ij}\} \,dG(\gamma) \,dF(k)}{\epsilon} \\
    & = \int_{0}^{\infty} \{ {s_{ij}} (\bar{\kappa_l}, \gamma)^{\star} - s_{ij} (\bar{\kappa_l}, \gamma)\} \,dG(\gamma) \,f(\bar{\kappa_l}) 
\end{align*}

As for $X_j$, the steps taken are almost the same, only that after raising the threshold by $\epsilon$ the amount of sick leaves granted goes around $\bar{\kappa_j}$ goes to $0$, thus giving out:
\[
\frac{\partial X_{j}}{\partial\bar{\kappa_{j}}} = - \int_{0}^{\infty}  s_{ij} (\bar{\kappa_j}, \gamma) \,dG(\gamma) \,f(\bar{\kappa_j})
\]
such that it is simply required that $j$ has positive demand for the mass of patients whose $\kappa_i = \bar{\kappa_j}$ for this derivative to be strictly negative, else it is null.

Likewise:
\[
\frac{\partial X_{j}}{\partial\bar{\kappa_{l}}} = - \int_{0}^{\infty}  s_{ij} (\bar{\kappa_l}, \gamma) \,dG(\gamma) \,f(\bar{\kappa_l})
\]

\end{proof}

Before dealing with PROPOSITION we need one final corollary.

\begin{corollary}
\[
\frac{\partial^2 Q_{j}}{\partial\bar{\kappa_{j}} \partial\bar{\kappa_{l}}} = 0 \;\;\; \text{ and } \;\;\; \frac{\partial^2 X_{j}}{\partial\bar{\kappa_{j}} \partial\bar{\kappa_{l}}} = 0 \;, \;\;\; \text{ $\forall j$ and $l$, $\bar{\kappa_{j}} \neq \bar{\kappa_{l}}$}
\]
 \end{corollary}

\begin{proof}
Our proof will be heuristic. Consider
 \[
    \frac{\partial Q_{j}}{\partial\bar{\kappa_{j}}} = \int_{0}^{\infty} \{ {s_{ij}} (\bar{\kappa_j}, \gamma)^{\prime} - s_{ij} (\bar{\kappa_j}, \gamma)\} \,dG(\gamma) \,f(\bar{\kappa_j})
 \]

Patient strategies $s_{ij}$ are taken at the atom $\kappa_i = \bar{\kappa_j}$. For any other physician $l$ such that $\bar{\kappa_l} \neq \bar{\kappa_j}$, an infinitesimally small change to $\bar{\kappa_l}$ doesn't affect the utility of patients at $\bar{\kappa_j}$, who will remain above or below $l$'s threshold same as before. Thus their strategy $s_{ij}$ also remains unaffected. If $\bar{\kappa_l}$ \textit{does} equal $\bar{\kappa_j}$, then the following limit is not properly defined:
\[
    \lim_{\epsilon \rightarrow 0} \frac{ \frac{\partial Q_{j}(\kappa_l + \epsilon)}{\partial\bar{\kappa_{j}}} - \frac{\partial Q_{j}(\kappa_l ) }{\partial\bar{\kappa_{j}}}   }{\epsilon}
\]
because as $\epsilon$ approaches $0$ from the right the limit is weakly positive ($\geq 0$), whereas approaching from the left it's strictly $0$. Meaning the derivative $\frac{\partial^2 Q_{j}}{\partial\bar{\kappa_{j}} \partial\bar{\kappa_{l}}}$ doesn't exist for $j$ and $l \neq j$ when $\bar{\kappa_j}=\bar{\kappa_l}$.

\end{proof}

PROOF OF PROPOSITION

Consider some two physicians $j$, $l$, such that $\bar{\kappa_j} \neq \bar{\kappa_l}$. If we breakdown the mixed partial derivative of $U_j$ over $\bar{\kappa_j}$ and $\bar{\kappa_l}$, we get:
\begin{align*}
\frac{\partial^2 U_{j}}{\partial\bar{\kappa_{j}} \partial\bar{\kappa_{l}}} = & 
\underbrace{R^{\prime \prime}(Q_j(\bar{\kappa_j}, \bar{\kappa}_{-j}))}_{\leq 0}
\cdot \underbrace{\frac{\partial Q_{j}}{\partial\bar{\kappa_{l}}}}_{\geq 0}
\cdot \underbrace{\frac{\partial Q_{j}}{\partial\bar{\kappa_{j}}}}_{\leq 0}  + 
\underbrace{R^{\prime} (Q_j(\bar{\kappa_j}, \bar{\kappa}_{-j}))}_{\geq 0} \underbrace{\frac{\partial^2 Q_{j}}{{\partial\bar{\kappa_{j}} \partial\bar{\kappa_{l}}}}}_{= \, 0} & \\
 & - \underbrace{P^{\prime \prime} (X_j(\bar{\kappa_j}, \bar{\kappa}_{-j}))}_{\geq 0}
\cdot \underbrace{\frac{\partial X_{j}}{\partial\bar{\kappa_{l}}}}_{\geq 0}
\cdot \underbrace{\frac{\partial X_{j}}{\partial\bar{\kappa_{j}}}}_{\leq 0}  - 
\underbrace{P^{\prime} (X_j(\bar{\kappa_j}, \bar{\kappa}_{-j}))}_{\geq 0} \underbrace{\frac{\partial^2 X_{j}}{\partial\bar{\kappa_{j}} \partial\bar{\kappa_{l}}}}_{= \, 0} &  \\
    = & \, \underbrace{{R^{\prime \prime}} (Q_j(\bar{\kappa_j}, \bar{\kappa}_{-j}))
    \cdot \frac{\partial Q_{j}}{\partial\bar{\kappa_{l}}}
    \cdot \frac{\partial Q_{j}}{\partial\bar{\kappa_{j}}}}_{\geq 0}  -
    \underbrace{P^{\prime \prime} (X_j(\bar{\kappa_j}, \bar{\kappa}_{-j}))
    \cdot \frac{\partial X_{j}}{\partial\bar{\kappa_{l}}}
    \cdot \frac{\partial X_{j}}{\partial\bar{\kappa_{j}}}}_{\leq 0} \, & \geq 0
\end{align*}

POOSITIVE MIXED DERIVATIVE IS STRATEGIC COMPLEMENTARITY

\subsection{Proof Sequential}

We follow the argument of (QUOTE). Let for the moment $\tilde{F_i}(U) = \operatorname{Pr}\{u_{ij} \leq U\}$ be the distribution of utility patient $i$ recieves from doctor visits, with $\tilde{F_i}(0) = 0$, $\tilde{F_i}(B) = 1$, for $B < \infty$. Each period, the patient draws an 'offer' $U$ from $\tilde{F_i}$, which is to say, his search lead him to a physician $j$ such that $u_{ij} = U$. He may either accept and have physician $j$ assigned to him, recieving utility $U$ per period henceforth, or look for a new physician next period.

Let $u_t$ be the patient's utility for period $t$, where it's $0$ if he doesn't visit a physician and $U$ if he has accepted an assigned physician which renders him utility $u_{ij} = U$. This patient will seek to maximize $\mathbb{E}[\sum_{t=0}^{\infty} \beta^t u_t]$, where $0 < \beta < 1$ is his discount factor.

Let $V(U)$ be the expected value of $\sum_{t=0}^{\infty} \beta^t u_t$ of a patient with an offer $U$ in hand. Assuming no recall, under optimal behavior the value function $V(U)$ satisfies the Bellman equation:
\begin{equation}
    V(U) = \max \left\{ \frac{U}{1 - \beta},\; \;\beta \hspace{-0.1cm}\int V(U') \, d \tilde{F_i}(U') \right\}
    \label{eq:seq1} 
\end{equation}

The patient chooses an optimal value $\bar{U_i}$, such that offers $U \geq \bar{U_i}$ are accepted (the visit takes place). As such, the solution takes on the following form:
\begin{equation}
    V(U) = \begin{cases}
        \frac{\bar{U}_i}{1 - \beta} + \beta \int V(U') \, d \tilde{F_i}(U') & \text{if }\; \; U \leq \bar{U}_i, \\[0.5em]
        \frac{U}{1 - \beta} & \text{if } \; \; U \geq \bar{U}_i.
    \end{cases}
\label{eq:seq2} 
\end{equation}

Using equation (\ref{eq:seq2}), we can convert the functional equation (\ref{eq:seq1}) into an ordinary equation in the reservation utility $\bar{U_i}$. Evaluating $V(\bar{U_i})$ and using equation (\ref{eq:seq2}), we have:

$$
\frac{\bar{U_i}}{1-\beta}= \beta \int_0^{\bar{U_i}} \frac{\bar{U_i}}{1-\beta} \, d \tilde{F_i}(U') + \beta \int_{\bar{U_i}}^B \frac{U'}{1-\beta} \, d \tilde{F_i}(U')
$$
or
$$
\begin{aligned}
& \frac{\bar{U_i}}{1-\beta} \int_0^{\bar{U_i}} d \tilde{F_i}(U') + \frac{\bar{U_i}}{1-\beta} \int_{\bar{U_i}}^B d \tilde{F_i}(U') \\
& \quad = \beta \int_0^{\bar{U_i}} \frac{\bar{U_i}}{1-\beta} \, d \tilde{F_i}(U') + \beta \int_{\bar{U_i}}^B \frac{U'}{1-\beta} \, d \tilde{F_i}(U')
\end{aligned}
$$
or
$$
\bar{U_i} \int_0^{\bar{U_i}} d \tilde{F_i}(U') = \frac{1}{1-\beta} \int_{\bar{U_i}}^B \left(\beta U' - \bar{U_i}\right) d \tilde{F_i}(U') .
$$

Adding $\bar{U_i} \int_{\bar{U_i}}^B d \tilde{F_i}(U')$ to both sides gives
\begin{equation}
    \bar{U_i} = \frac{\beta}{1-\beta} \int_{\bar{U_i}}^B \left(U' - \bar{U_i}\right) d \tilde{F_i}(U')
    \label{eq:seq3}  
\end{equation}


Finally, for our purposes, we discretize $\int_{\bar{U_i}}^B \left(U' - \bar{U_i}\right) d \tilde{F_i}(U')$ as a summation over the countable set of $J$ doctors. The discrete equivalent of equation (\ref{eq:seq3}) is:
\begin{equation}
    \bar{U_i} = \frac{\beta}{1-\beta}  \sum_{j=1}^{J} \left\{ \frac{\mathbbm{1}[ u_{ij} \geq \bar{U_i} ] \cdot (u_{ij} - \bar{U_i})}{\mathbbm{1}[ u_{ij} \geq \bar{U_i} ]} \right\}
    \label{eq:seq4}  
\end{equation}

Which is the function we say characterizes patient thresholds in the explicit search model.


\end{document}
