\documentclass[11pt]{article}

\usepackage[letterpaper,left=4cm,right=4cm,top=3cm,bottom=3cm]{geometry}

\usepackage{graphicx} % Required for inserting images
\usepackage{amsmath, amsthm, amssymb}
\usepackage{bbm} %para 1 indicatriz
\usepackage{cancel} %strike symbols
\usepackage{enumitem} %fancy itemize
\usepackage{thmtools} %fancy theorems
\usepackage{bm} %math bold-face
\usepackage{float} %To have [H] in figure
\usepackage{subcaption} % For subfigure environment
\usepackage{caption} % To space captions
\captionsetup{justification=centering}
\usepackage{footmisc}  % For easier footnote referencing

\usepackage{pgfplots} % Tikz package
\pgfplotsset{compat=1.18}
\usetikzlibrary{arrows.meta}


\usepackage{xltabular} % to span multiple pages

\usepackage[dvipsnames]{xcolor} %better colors
\usepackage{natbib} %bibliography style
\setlength{\bibsep}{5.25pt}

\usepackage[colorlinks=true,
        linkcolor=black,
        citecolor=Sepia,
        urlcolor=black]{hyperref}

\usepackage{subfiles} %for organization

\usepackage[T1]{fontenc}
\usepackage{baskervillef}

\usepackage{booktabs}

\hyphenpenalty=1000 %Higher hyphenation penalty

\linespread{1}
\setlength{\parskip}{5pt}
\setlength{\footnotesep}{.2in}
\setlength{\parindent}{10pt}

\DeclareMathOperator*{\argmax}{arg\,max} %Define argmin and argmax
\DeclareMathOperator*{\argmin}{arg\,min}

\newtheorem{theorem}{Theorem}[section]
\newtheorem{lemma}[theorem]{Lemma}
\newtheorem{corollary}[theorem]{Corollary}
\newtheorem{definition}[theorem]{Definition}
\newtheorem{ass}[theorem]{Assumption}
\newtheorem{prop}[theorem]{Proposition}

\newcommand\maybegeq{\stackrel{?}{\geq}}

\declaretheoremstyle[
notefont=\bfseries, notebraces={}{},
bodyfont=\normalfont\itshape,
headformat=\NAME \NOTE
]{nopar}
\declaretheorem[style=nopar]{equilibrium}

\renewcommand{\thesubsubsection}{\alph{subsubsection}}
% Subsubsections with .a .b etc

\title{\includegraphics[scale=0.3]{UC_lineal_TR-01.png} \linebreak \linebreak 
Master Thesis in Economics \linebreak \linebreak 
{\fontsize{22pt}{50pt}\selectfont\textit{Search and Competition \vspace{10pt}\linebreak  in the Market for Sick Leaves \vspace{20pt}}}}

\author{León Tagle Le Blanc \\
Advisors: Prof. Nicolás Figueroa, Prof. Pablo Celhay}
\date{December 16th, 2024}

\begin{document}

\begin{titlepage}
\maketitle
\thispagestyle{empty}

\vspace{4em}

\begin{abstract}
        This paper develops a market framework for sick leaves between physicians and patients, where patients imperfectly search for physicians based on their heterogenous quality and endogenous issuance leniency. This framework rationalizes the effects of competition on sick leave issuance as well as the assortative matching between ``dishonest'' physicians and patients. Using data on sick leaves from Chile's national insurer, a specification of this framework is estimated and counterfactual scenarios are run. We find that a 64\% rise in the fine associated to sick leave issuance would reduce aggregate sick leaves in the market by 32\%.
\end{abstract}

\newpage
\tableofcontents
\setcounter{page}{0}
\thispagestyle{empty}

\end{titlepage}

\setlength{\parskip}{6pt}

\subfile{Introduction/introduction}

\subfile{Literature/literature}

\subfile{Model/model}

\subfile{Computation/computation}

\newpage

\subfile{Data/data}

\subfile{Estimation/estimation}



\newpage



\bibliography{bibliography}
\addcontentsline{toc}{section}{References}	 
\bibliographystyle{dcu}

\newpage

\subfile{Appendix/appendix}


\end{document}