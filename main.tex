\documentclass[11pt]{article}

\usepackage[letterpaper,left=4cm,right=4cm,top=3cm,bottom=3.5cm]{geometry}

\usepackage{graphicx} % Required for inserting images
\usepackage{amsmath, amsthm, amssymb}
\usepackage{bbm} %para 1 indicatriz
\usepackage{cancel} %strike symbols
\usepackage{enumitem} %fancy itemize
\usepackage{thmtools} %fancy theorems
\usepackage{bm} %math bold-face
\usepackage{float} %To have [H] in figure
\usepackage{subcaption} % For subfigure environment
\usepackage{caption} % To space captions
\captionsetup{justification=centering}

\usepackage[dvipsnames]{xcolor} %better colors
\usepackage{natbib} %bibliography style
\usepackage[colorlinks=true,
        linkcolor=black,
        citecolor=Sepia,
        urlcolor=blue]{hyperref}

\usepackage{subfiles} %for organization

\usepackage[T1]{fontenc}
\usepackage{baskervillef}

\usepackage{booktabs}

\hyphenpenalty=500 %Higher hyphenation penalty

\linespread{1}
\setlength{\parskip}{6pt}
\setlength{\footnotesep}{.2in}
\setlength{\parindent}{0pt}

\DeclareMathOperator*{\argmax}{arg\,max} %Define argmin and argmax
\DeclareMathOperator*{\argmin}{arg\,min}

\newtheorem{theorem}{Theorem}[section]
\newtheorem{lemma}[theorem]{Lemma}
\newtheorem{corollary}[theorem]{Corollary}
\newtheorem{definition}[theorem]{Definition}
\newtheorem{ass}[theorem]{Assumption}
\newtheorem{prop}[theorem]{Proposition}

\newcommand\maybegeq{\stackrel{?}{\geq}}

\declaretheoremstyle[
notefont=\bfseries, notebraces={}{},
bodyfont=\normalfont\itshape,
headformat=\NAME \NOTE
]{nopar}
\declaretheorem[style=nopar]{equilibrium}

\renewcommand{\thesubsubsection}{\alph{subsubsection}}
% Subsubsections with .a .b etc

\title{\includegraphics[scale=0.3]{UC_lineal_TR-01.png} \linebreak \linebreak 
Thesis Draft \linebreak Master in Economics}
\author{León Tagle Le Blanc}
\date{as of 02-nov-2024}

\begin{document}

\begin{titlepage}
\maketitle
\thispagestyle{empty}

\newpage
\tableofcontents
\setcounter{page}{0}
\thispagestyle{empty}
\end{titlepage}


\subfile{Introduction/introduction}

\subfile{Literature/literature}


\subfile{Model/model}

\subfile{Computation/computation}

\section{Data}

We use administrative data provided by FONASA from all sick leaves issued from January 2018 to October 2022 in the public health system.

Overall totals by patients and physicians in the sampled period have the following descriptive statistics.

\begin{table}[h]
    \centering
    \begin{tabular}{lrr}
    \toprule
    Statistic & Physicians & Patients \\
    \midrule
    Mean      & 88     & 3       \\
    Std       & 199     & 3       \\
    Min       & 1      & 1      \\
    Max       & 8,743    & 73     \\
    Percentiles & & \\
    \hspace{1em}10th     & 2       & 1       \\
    \hspace{1em}50th  & 24    & 2       \\
    \hspace{1em}90th    & 216     & 7       \\
    99th     & 1007    & 14      \\
    Count     & 48,611   & 1,445,696 \\
    \bottomrule
    \end{tabular}
    \caption{Summary Statistics for Physicians and Patients}
\end{table}




\section{Calibration}

The first step required in calibration is to divy up the physicians into bins. The way this is achieved by by setting up a three part 'quality' score, composed of the physician's percentile in three categories: average days of sickness leave granted, percentage of sick leaves issued over 30 days, amount of sick leaves issued over 30 days.

The logic behind these categories can be surmissed in the models section. As physician quality ($V_j$) increases, so too their choice of threshold ($\bar{\kappa_j}$) in order to avoid excessive patient demand. As a result, the composition of their patient-base tends to a higher average $\kappa_i$, including a higher percentage of $\kappa_i \geq \bar{\kappa}_{\max}$. Sickness leave of $\geq 30$ days is subject to higher scrutiny and requires special approval, as well as being indicative of serious illness/injury. As such, issuances of $\geq 30$ days, composing close to $21\%$ of our sample, are taken to be such given to patients whose $\kappa_i \geq \bar{\kappa}_{\max}$, and a higher percentage of those in $Q_j$ taken to be indicative of higher $V_j$.

Taking the average percentile in the three categories mentioned above, we set up four bins of physician quality, with the following frequency in the sample

\begin{table}[H]
\centering
\begin{tabular}{lc}
\toprule
 & \% \\
    $V_{high}$  &   39.65 \\
    $V_{mid-high}$ & 21.69 \\
    $V_{mid-low}$ & 20.00 \\
    $V_{low}$   & 18.66 \\
\bottomrule 
\end{tabular}
\caption{Frequency of physician quality $V_j$ bins}
\end{table}

We perform calibration of the Logit model. We perform GMM on 9 sample moments as can be seen below. We use the subsample of day sick leaves because it allows for linear independence from the parameters determining $G(\gamma)$, as a patient with $\kappa_i \geq \bar{\kappa}_{\max}$ will be granted sick leave by any physician, such that:
\[
    s_{ij} = \frac{\alpha_{ij}}{\sum_{k = 1}^{J} \alpha_{ik}} =
    \frac{e^{\lambda u_{ij}}}{\sum_{k = 1}^{J} e^{\lambda u_{ik}}} =
    \frac{e^{\lambda \{V_j \kappa_i + \cancel{\gamma_i} - \tau_j \}}}{\sum_{k = 1}^{J} e^{\lambda \{V_k \kappa_i + \cancel{\gamma_i} - \tau_k \}}} =
    \frac{e^{\lambda \{V_j \kappa_i + - \tau_j \}}}{\sum_{k = 1}^{J} e^{\lambda \{V_k \kappa_i + - \tau_k \}}}
\]

We assume $F(\kappa)$ to be an exponential function where we normalize the shape parameter $\lambda_F$ (subcscript F, subscript S for the model one) to 1. $G(\gamma)$ is a normal distribution whose parameters we'll seek to shape. We also look to calibrate the value of each of the four bins, and finally the value of $\kappa_{\max}$. We perform a separate GMM prior to the main one to get the parameters determining $P(\cdot) = \text{fine} * \operatorname{Pr}[\text{fine} | X_j] = t * (\theta X_j)^2$.

\begin{table}[h]
\centering
\begin{tabular}{lcc}
    \toprule
    Parameter & Value & Source \\
    \midrule
    Quality bins && \\
        \hspace{1em}$V_{high}$  &   - & 2nd GMM \\
        \hspace{1em}$V_{mid-high}$ & - & 2nd GMM\\
        \hspace{1em}$V_{mid-low}$ & -& 2nd GMM \\
        \hspace{1em}$V_{low}$   & - & 2nd GMM\\
        \\
    Punishment function && \\
        \hspace{1em}$t$ & - & 1st GMM\\
        \hspace{1em}$\theta$ & - & 1st GMM\\
        \\
    Medical need distribution $F(\kappa)$ && \\
        \hspace{1em}$\lambda_F$ & - & Normalization\\
        \\
    Taste distribution $G(\gamma)$ \\
        \hspace{1em}$\mu$ & - & 2nd GMM\\
        \hspace{1em}$\sigma$ & - & 2nd GMM\\
        \\
    Maximum threshold level && \\
        \hspace{1em}$\bar{\kappa}_{\max}$ & - & 2nd GMM\\
        \\
    Logit model parameter && \\
        \hspace{1em}$\lambda_s$ & - & 2nd GMM\\
        \\
    Revenue/cost of visit && \\
        \hspace{1em}$r_j$ & - & Data \\
        \hspace{1em}$\tau_j$ & - & Data \\
    \bottomrule 
\end{tabular}
\caption{Model parameters}
\end{table}

Data calibration relies on nine moments: sick leaves issued by bin, sick leaves over 30 days by bin, proportion of $\geq30$ in total.

\begin{table}[h]
    \centering
    \begin{tabular}{lcc}
        \toprule
        Moment & Data & Simulated \\
        \midrule
        Total sick leaves issued && \\
            \hspace{1em}$V_{high}$ & 967,956 & -  \\
            \hspace{1em}$V_{mid-high}$& 1,055,544 & - \\
            \hspace{1em}$V_{mid-low}$& 1,068,364& -  \\
            \hspace{1em}$V_{low}$ &  1,481,964& -\\
            \\
        Sick leaves issued of $\geq 30$ days && \\
            \hspace{1em}$V_{high}$& 656,920 & -  \\
            \hspace{1em}$V_{mid-high}$& 239,168 & - \\
            \hspace{1em}$V_{mid-low}$& 53,270& -  \\
            \hspace{1em}$V_{low}$ & 4,832 & -\\
            \\
        Proportion of $\geq 30$ days sick leaves over total & 2,086& - \\
        \bottomrule 
    \end{tabular}
    \caption{Model parameters}
    \end{table}

\section{Counterfactuals \& Comments}

PENDING.

\newpage


\bibliography{bibliography}
\addcontentsline{toc}{section}{References}	 
\bibliographystyle{dcu}

\newpage

\subfile{Appendix/appendix}


\end{document}