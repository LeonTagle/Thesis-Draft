\documentclass[../main.tex]{subfiles}

\begin{document}

\section{Introduction}

The question of sickness at work presents us with a principal-agent problem: though in cases of true sickness it is best for the employee's personal wellbeing and indeed even firm productivity \citep{productivity} that they should stay home, it would also be possible to call in sick without any affliction, skipping work undetected.

The common solution to this problem is third-party certification, a specialized professional to attest to the condition and allow for or even demand of the employer that sick leave be granted. And who else to carry out this evaluation than physicians themselves—knowledgeable on the subject as they are—making diagnosis, treatment referral and certification of sickness a one-stop-shop for the patients.

The problem with this oversized, multi-faceted role physicians play in healthcare systems is that it causes a conflict of interests. In cases where sick leave (and insurance claims) call for state-subsidied payments to be issued, we can speak of a second principal-agent problem between the physician, exercising the role of ``gatekeeper'', and the taxpayer, bankrolling public healthcare and insurance.

There is evidence to the fact that even in the best of cases physicians have no incentive to place doubt on their patients claim to unobservable symptoms \citep{cln}, and in murkier scenarios they could be motivated to gain the reputation of being ``lenient'' to enjoy increased demand for their services \citep{markussen-roed}.

In Chile, the country of our sample, the healthcare system is bi-modal: citizens may opt for one of many private health insurers (ISAPRES) or the sole public option (FONASA), and 82\% of them are affiliates of the latter. In their case, the authority behind sick leave granting is COMPIN, with the power to oversee, investigate and even sanction physicians that fall under it. A worker granted sickness absence is entitled to a work incapacity benefit (SIL), periodic payments paid in function of sick leave length.

Anecdotal evidence speaks to the presence of fraud in the system. A survey by the U. Andrés Bello Public Health Institute (\citeyear{unab}) has it that 51\% of those surveyed know someone who was granted sick leave without any sickness, 62\% believe that physicians ``frequently'' create irregular businesses for the sale of sick leave certificates, and 56\% think it would be ``easy'' to purchase one.

The presence of fraud is not unknown to the overseeing authority. In September 2021, COMPIN sanctioned 188 physicians who were not able to justify their unusually high issuance rate. \cite{oteiza} finds that this one time intervention had a spillover on non-sanctioned doctors, who on average reduced their issuance rate by at least 5.22\%, on conservative estimates. This is further proof that sick leave granting \textit{is} influenced by profit-seeking, and readjusted expectations of the probability of being met with punishment does influence physicians' willingness to provide them to patients.

This paper seeks to develop a framework within which physicians' behavior may be rationalized and modeled. It presents physicians in a dual role as both medical caretakers and issuers of sick leave, and patients value both aspects separately according to population distributions. Patient strategies are the probabilities with which they'll visit any given physician, and physician strategies is their choice of ``leniency'', how willing they are to grant sick leave according to patient characteristics.

A generalized framework of patient search is laid out and contrasted with a non-search baseline, i.e. a particular kind of equilibrium in which patient strategies are set as randomized over all physicians with no distinction, that is to say, in which they're assigned a physician with no say on the matter but whether to visit. Once patient `search' comes into play, physicians take into consideration the behavior of the rest of their colleagues—and its implication on their own market share—when opting for a given level of strictness in sick leave issuance, meaning there's a \textit{strategic} element now involved, where physicians will be wary to lose out on too many patients to more \textit{lenient} competitors.

We then present two particular specifications of the `search' framework: an ``implicit'' search model, where patient behavior takes on the form of a modified McFadden Logit,\footnote{See \cite{mcfadden}.} and an ``explicit'' search model, akin to sequential job search models, in which patients visit physicians which render them services above their ``reserve utility''.

The dual sources of patient utility—services and sick leave—combined with the strategic nature of the game, where each physician's strategy responds to the choices of all other physicians, render a closed form equilibrium equation infeasible, requiring numerical methods. We develop best-response algorithms to compute the various equilibrium strategies and market aggregates.

Finally, using data on sick leave issuance from FONASA, Chile's national public insurer, we fit our model through a GMM exercise and run a counterfactual scenario. Being impossible for a policymaker to reliably police physician issuance to patients who don't require sick leave, as for conditions of low severity there's little scrutiny to their diagnosis, we compute the magnitude of the fine required so that, at the very least, \textit{aggregate} sick leave granting be at the level it would be if direct surveillance were possible.

Our result is that if the perceived probability of a fine or its magnitude were to increase around $1.64x$ at each level of sick leaves issued, physicians would endogenously adjust their strictness so that aggregate sick leave granting in the physician-patient market correspond to the level it would have if the policymaker could specifically prevent issuance to the 32\% of patients of least severity, including those with minimal or no symptomatology (where sick leave issuance constitutes fraud).

The empirical element of this paper, estimation, is somewhat hamstrung by the data available, chiefly the lack of information on overall visits to physicians rather than just the sick leaves they have issued and of more observables on physician that could allow for an independent assessment of their ``quality''. Possible, or indeed needed extensions to this work, if it should prove useful to policymaking, should begin with refining this aspect. Fitting the model thorugh GMM would greatly benefit from more varied and reliable data moments, as well as having more of the parameters involved be retreived from the data itself.

The rest of the paper proceeds as follows. Section 2 presents a brief review of the literature concerning physician markets, particularly those involving \textit{prescriptions}. Section 3 presents the general theoretical framework of our model as well as two specifications. Section 4 addresses the numerical computation of equilibrium in these models. Our available data on sick leave issuance is covered in Section 5, which will be used in Section 6 to fit our model, calibrating the required parameters through GMM. Section 7 discusses policy, increased penalization for physicians, with a simulated counterfactual, and Section 8 concludes.

\end{document}