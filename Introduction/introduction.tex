\documentclass[../main.tex]{subfiles}

\begin{document}

\section{Introduction}

The question of sickness at work presents us with a principal-agent problem: though in cases of true sickness it is best for the employee's personal wellbeing and indeed even firm productivity (FUENTE) that they should stay home, it would also be possible to call in sick without any affliction, skirting work undetected.

The common solution to this problem is third-party certification, a specialized professional to attest to the condition and allow for or even demand of the employer that sick leave be granted. And who else to carry out this evaluation than physicians themselves, knowledgeable on the subject as they are, making diagnosis, treatment referral and certification of sickness a one-stop-shop for the patients.

The problem with this oversized, multi-faceted role physicians play in healthcare systems is that it causes conflict of interests. In cases where sick leave calls (and insurance claims) call for state-subsidied payments to be issued, we can speak of a second principal-agent problem between the physician, exercising the role of ``gatekeeper'', and the taxpayer, bankrolling public healthcare and insurance.

There is evidence to the fact that even in the best of cases physicians have no incentive to place doubt on their patients claim to unobservable symtoms (FUENTE), and in murkier scenarios they could be motivated to gain the reputation of being ``lenient'' to enjoy increased demand for their services (MARKUSSEN).

In Chile, the country of our sample, the healthcare system is bi-modal: citizens may opt for one of many private health insurers (ISAPRES) or the sole public option (FONASA), and 82\% of them are affiliates of the latter. In their case, the authority behind sick leave granting is COMPIN (fuente?), with the power to oversee, investigate and even sanction physicians that fall under it. A worker granted sickness absence is entitled to a work incapacity benefit (SIL), periodic payments paid in function of sick leave length (fuente?).

Anecdotal evidence speaks to the presence of fraud in the system. A survey by the U. Andrés Bello Public Health Institute (YEAR) has it that 51\% of those surveyed know someone who was granted sick leave without any sickness, 62\% believe that physicians ``frequently'' create irregular businesses for the sale of sick leave certificates, and 56\% think it would be ``easy'' to purchase one.

EXPERIMENTO?



\end{document}