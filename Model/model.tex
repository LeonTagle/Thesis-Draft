\documentclass[../main.tex]{subfiles}

\begin{document}

\section{Model}

We now present the model itself before proceeding with further discussion. We consider $i = 1, ..., I$ patients and $j = 1, ..., J$ physicians.\footnote{Note on language: For ease of reference, we refer to any patient as ``he'', and to any physician as ``she'' (s we have done already). At this point it shall be noted that the first person plural (we) employed over the course of this article is to be read as including the reader or perhaps as a royal \textit{we}.}

Patients are characterized by their ``medical necesity'' $\kappa_i \in \mathbb{R}_0^+$ and their ``taste for sick leave'' $\gamma_i \in \mathbb{R}_0^+$, respectively following the ex-ante cumulative distributions $F(\kappa)$ and $G(\gamma)$, which are public knowledge. Physicians are described by their ``service quality'' $V_j \in \mathbb{R}^+$, also known both to all patients and other physicians.

A patient $i$ can visit a physician for treatment and may be also granted sick leave. After the patient is assigned a physician $j$, his utility function—implicitly dependent on his characteristic $(\kappa_i,\gamma_i)$ tuple—is defined piecewise as follows:
    \[
U_i(V_j) =\begin{cases}
\gamma_i + V_j \kappa_i - \tau_j \text{  \hspace{0.65cm} if he’s granted sick leave,} \\
V_j \kappa_i - \tau_j \text{  \hspace{0.5cm} if he only visits the physician,} \\
0 \text{  \hspace{0.8cm} if no visit takes place}
\end{cases}
\]

As we see, there are three components to patient utility: an interaction between the patient's medical need $\kappa_i$ and the physician's service quality $V_j$ which implies their complimentarity, his ``taste'' for sick leave $\gamma_i$ in the case he's granted one, and $\tau_j$, which we define as the cost of visit for physician $j$. As well as being complementarity, $\kappa_j$ as multiplied by $V_j$ would give it the interpretation of being the marginal willigness to pay for physician quality.\footnote{Conceptually, willingness to pay would also depend on patient $i$'s wealth level $w_i$. To steer away from discussions on wealth disparity, we interpret $\kappa_i$ as the \textit{average} willingness to pay for $V_J$.}

Whereas patients may visit at most one physician, a physician may see several patients. We define $Q_j$ as the expected number of patients for physician $j$, the demand for her services. We say `expected' because, as we'll see later on, patients may opt for a mixed strategy, assigning a certain \textit{probability} to visiting $j$, and we define $Q_j$ as the sum of the ex-ante probability of visit of all patients, not their ex-post realization.

As physician $j$ has the option to issue sick leave to a given patient $i$ which visits her, we likewise define $X_j$ as the \textit{expected} number of sick leaves granted by physician $j$, given her ex-ante patient demand and how many of them would be granted one.

We define the physician's utility function as follows:
\[
U_j(Q_j, X_j) = R_j(Q_j) - P(X_j)
\]
where $R_j(\cdot)$ is an individual, concave \textit{revenue} function defined over expected total clients, and $P(\cdot)$ is a convex \textit{punishment} function on $X_j$, composed of the probability of being fined for a certain number of sick leaves issued and the magnitude of the fine. The implication is that after a given number of patients, the disutility of an additional (expected) sick leave issued would outweigh physician $j$'s financial incentive for further clientele. The $j$ subindex indicates that we will allow $R_j(\cdot)$ (revenue by visit) to vary across physicians. We assume both $R_j(\cdot)$ and $P(\cdot)$ to be twicely differentiable.

We stress again that we define physician utility in terms of the \textit{expected} realization of patient demand and granted sick leaves, as befits the logic of this game, where doctors take action before patients (we will specify the timing of our game below).

Following \cite{schnell2017physician}, we focus on \textit{threshold equilibria}, wherein each physician's strategy is the choice of a value $\bar{\kappa_j} \in \mathbb{R}_0^+$, such that of the patients who visit $j$, those with a $\kappa_i$ value above or at that threshold will be granted sick leave, and those strictly under it won't. As a result of this, each physician will be identified by their given `service quality', revenue function, cost of visit to the patient and choice of threshold: $\{V_j,R_j(\cdot), \tau_j, \bar{\kappa_j}\}$. The set $\{\bar{\kappa_j}\}_{j=1}^{J}$ will be known to patients when deciding on their behavior strategy.

Our models are different iterations of a general game with the following timing:
\begin{itemize}[itemsep=-1pt, topsep=0pt]
    \item First stage: Physicians simultaneously choose $\bar{\kappa_j}$.
    \item Second stage: Observing $\{V_j,R_j(\cdot), \tau_j, \bar{\kappa_j}\}$. The set $\{\bar{\kappa_j}\}_{j=1}^{J}$, each patient chooses (or is assigned to) some doctor $j$.
    \item Third stage: Each patient can choose to see their physician and incur a visit cost $\tau_j$, or refrain from doing so.
\end{itemize}

Conditional upon his visit (as not to visit renders null utility), the utility of patient $i$ from seeing physician $j$ is:
\[
    U_i(V_j,\bar{\kappa_j}) =\begin{cases}
    \gamma_i + V_j \kappa_i - \tau_j \text{  \hspace{0.65cm} if $\kappa_i \geq \bar{\kappa_j}$,} \\
    V_j \kappa_i - \tau_j \text{  \hspace{0.5cm} if $\kappa_i < \bar{\kappa_j}$}
    \end{cases}
\]

We will usually use $u_{ij}$ as a shorthand.

\subsection{Non-search Equilibrium}

For illustration purposes, we first devote attention to a non-search baseline, where each patient is randomly assigned to a physician, with an equal probability of being matched to any of the $J$ physicians. Their only say in the matter is whether they'll then visit physician $j$, i.e. the second stage of the game is out of their hands, and they only make choices in the third stage after assignment.

A patient won't visit his assigned physician if his expected utility from such a visit is negative, we call this a \textit{free disposal} requirement. As such, a doctor $j$’s expected client demand, as a function of $\bar{\kappa_j}$ and given the parameter $V_j$, will be the following:
\begin{equation}
    Q_j(\bar{\kappa_j}) \,=\, \frac{I}{J}\left[ \int_{\tau/V_j}^{\infty}\,dF(k) +  \int_{\min\{\bar{\kappa_j},\tau/V_j\}}^{\tau/V_j} \int_{\tilde{\gamma}(k)}^{\infty} \,dG(\gamma) \,dF(k) \right] \tag{N.1}\label{eq:ns_Q}
\end{equation}
where the left term consists of the mass of patients who just by virtue of doctor $j$’s service quality $V_j$ would be willing to pay a visit (i.e. $\kappa_i \geq \tau/V_j$), and the right term would be patients who only see doctor $j$ out of the expectation of getting sick leave ($\kappa_i \geq \bar{\kappa_j}$ \& $\gamma_i \geq \tau - V_j \kappa_i$), but wouldn’t visit otherwise. We define $\tilde{\gamma}(k) := \tau - V_j k$ as the lower limit of the inner integral.

Given that each patient with a $\kappa_i$ higher or equal to $\bar{\kappa_j}$ is granted a sick leave certificate, the expected total number of such certificates granted by $j$, as a function of $\bar{\kappa_j}$, is:

\begin{equation}
    X_j(\bar{\kappa_j}) \,=\, \frac{I}{J} \int_{\bar{\kappa_j}}^{\infty} \int_{\tilde{\gamma}(k)}^{\infty} \,dG(\gamma) \,dF(k)
    \tag{N.2}\label{eq:ns_X}
\end{equation}

Given (\ref{eq:ns_Q}) and (\ref{eq:ns_X}), each physician solves for the following unconstrained optimization:
\begin{equation}
\bar{\kappa_j}^* \equiv \operatorname{arg}\max_{\bar{\kappa_j}} R_j(Q_j) - P_j(X_j)
\tag{N.3}\label{eq:ns_k}
\end{equation}

We now SEE LATER

\begin{lemma}
\label{ns_lemma}
No value in $(\frac{\tau}{V_j},\infty)$ can be the optimal solution of (\ref{eq:ns_k}).
\end{lemma}

PROOF: APPENDIX

\begin{lemma}
    \label{ns_derivatives}
When $\bar{\kappa_j} \in [0,\frac{\tau}{V_j}]$,
\[
    \frac{\partial Q_j}{\partial\bar{\kappa_j}} = \frac{\partial X_j}{\partial \bar{\kappa_j}}   
\]
\end{lemma}

PROOF: APPENDIX

This proposition implies that when $\bar{\kappa_j}$ is low enough that the marginal patient is indifferent, any new clients gained by doctor $j$ are those she entices through the expectation of getting a certificate, meaning that in the vicinity of $\bar{\kappa_j}$ the change in expected patients $\Delta Q_j/\Delta\bar{\kappa_j}$ through a variation in $\bar{\kappa_j}$ is the same as the change in expected certificates granted $\Delta X_j/\Delta \bar{\kappa_j}$.

Given Proposition \ref{ns_derivatives}, the following equation, which is the FOC of (\ref{eq:ns_k}):
\[
R_j^{\prime}(Q_j)\frac{\partial Q_j}{\partial\bar{\kappa_j}}  - P_j^{\prime}(X_j)\frac{\partial X_j}{\partial \bar{\kappa_j}} = 0
\]
may be simplified into
\begin{equation}
    R^{\prime}(Q_j) = P^{\prime}(X_j)
    \tag{N.4}\label{eq:ns_FOC}
\end{equation}

We can see that (\ref{eq:ns_k}) offers three possibilities by way of solution: a \textit{corner} solution towards $\infty$, where doctor $j$ is content with her ``captured'' clientele (those who would visit her even without expecting sick leave), and offers no certificates at all\footnote{We may rationalize this by interpreting it as physician $j$ being as strict as is medically responsible. We could have written this as $\bar{\kappa_j}$ having an upper bound $\kappa_{\operatorname{max}}$, where any physician would be obliged to grant sick leave to a patient with $\kappa_i > \kappa_{\operatorname{max}}$.}; the corner solution $\bar{\kappa_j} = 0$, maximum leniency; and an \textit{inner} solution in $[0,\frac{\tau}{V_j}]$ fulfilling equation (\ref{eq:ns_FOC}) and thus the FOC of (\ref{eq:ns_k}). We formalize equilibrium below.

\begin{equilibrium}[in non-search models]
    \label{ns_eq}
Given a doctor-patient market specified by \\ $(\{(\kappa_i,\gamma_i)\}_{i =1}^{I},\{(V_j,R_j(\cdot),P_j(\cdot))\}_{i =1}^{J})$, we define an equilibrium as a set of physician thresholds $\{\bar{\kappa_j}\}_{j =1}^{J}$ satisfying (\ref{eq:ns_k}) and their corresponding equilibrium (expected) patient demand and granted sick leave certificates $\{(Q_j,X_j)\}_{j =1}^{J}$ as described by (\ref{eq:ns_Q}), (\ref{eq:ns_X}), respectively.
\end{equilibrium}

\subsection{Search Equilibria}

\subsubsection{Introducing search}

We introduce patient ``search'' as a general framework in which patients can choose freely among all physicians.

We define for each patient $i$ a vector $S_i \in \Delta(\mathcal{J})$, where $\mathcal{J}$ is the the $1 \times J$ vector composed of all $1, ..., J$ physicians. $S_i$ will be his \textit{strategy} for this game, representing his probabilistic choice of visit for each doctor $j$, such that each component $s_{i1}, ... , s_{iJ}$ of $S_i$ stands for the probability that he'll visit doctors $1, ..., J$ respectively.

In order to describe a proper probability distribution, the following criteria must be met:


\begin{enumerate}[label=\roman*.]
    \item $\forall j, \, s_{ij} \geq 0$
    \item $\sum_{i = 1}^{J} s_{ij} \leq 1$
\end{enumerate}

We will allow the sum of all components to be less than one, implying the presence of an \textit{outside option} for patients, that is, to not visit any doctors. Such an option is important, as patient rationality in our models will include  ``\textit{free disposal}'', meaning that a patient will never visit a doctor if his expected utility from such a visit is less than $0$, i.e. $s_{ij} = 0$ if $U_i(V_j,\bar{\kappa_j}) = 0$. This makes the third stage trivial, as under free choice, patient $i$ will only be assigned with positive probability to doctors he will be willing to visit.

We can re-interpret the non-search model as each patient being made to play by the strategy \{$S_i: s_{i1} = s_{i2} ... = s_{iJ} = \frac{1}{J}$ \}, and it's the lack of a free choice which makes the third stage non-trivial.

Given the collection of all patients' strategies $(\{S_i\}_{i =1}^{I}$, each with a respective $s_{ij}$ for doctor $j$, we can formulate expected clientele and granted sick leaves for $j$ in a manner very much alike the previous section (indeed, as a generalization):
\begin{equation}
    Q_j(\bar{\kappa_j}, \bar{\kappa}_{-j}) \,=\, \int_{\tau/V_j}^{\infty}\bm{s_{ij}}\,dF(k) +  \int_{\min\{\bar{\kappa_j},\tau/V_j\}}^{\tau/V_j} \int_{\tilde{\gamma}(k)}^{\infty}\bm{s_{ij}} \,dG(\gamma) \,dF(k) 
    \tag{S.1}\label{eq:s_Q}
\end{equation}

\begin{equation}
    X_j(\bar{\kappa_j}, \bar{\kappa}_{-j}) \,=\, \int_{\bar{\kappa_j}}^{\infty} \int_{\tilde{\gamma}(k)}^{\infty}\bm{s_{ij}} \,dG(\gamma) \,dF(k)
    \tag{S.2}\label{eq:s_X}
\end{equation}

As before, the left term in equation (\ref{eq:s_Q}) includes patients who would visit $j$ even without sick leave, whereas the right term those who only do so because they expect one.

Notice that we now present $Q_j$ and $X_j$  not only in terms of the threshold $\bar{\kappa_j}$ chosen by doctor $j$, but also in terms of the thresholds of the other $J - 1$ doctors, which we abreviate as $\bar{\kappa}_{-j}$. Whereas before doctors were simply alloted a given number of patients, now they will \textit{compete} for them, as patients' $S_i$ strategy will consider the whole of $(\{\bar{\kappa_j}\}_{j =1}^{J}$ when considering which physician(s) they will visit with positive probability.

\subsubsection{Some analytic results}

We shall further specify the form $S_i$ will take. Consider now a $1 \times J$ vector $U_i$, where each component $u_{i1}, ... , u_{iJ}$ indicates the utility patient $i$ expects from a visit to doctors $1, ..., J$ respectively, corresponding to $U_i(V_j, \bar{\kappa_j})$ for each doctor $j$. As a shorthand, we shall write $u_{i,-j}$ to indicate the $J - 1$ components of $U_i$ excluding $u_{ij}$.

In the models considered, each component $s_{ij}$ of $S_i$ will be defined as:
\[s_{ij} \equiv g_i(u_{ij}, u_{i,-j})\]
where $g_i(\cdot)$ is a continuous function weakly increasing in the first argument $u_{ij}$, and weakly decreasing in the remaining arguments given by $u_{i,-j}$. Our model specifications will consist in giving this function $g_i(u_{ij}, u_{i,-j})$ a specific form.

As we show in the APPENDIX, such prerrequisites over $s_{ij}$ are enough to prove most of the qualitative relations we expect to see regarding equilibrium aggregates and doctor strategies, but not enough to prove that $\bar{\kappa_j}$ and $\bar{\kappa}_{-j}$ are strategic complements -- i.e., that for any $l \neq j$, $\frac{\partial^2 U_{j}}{\partial\bar{\kappa_{j}} \partial\bar{\kappa_{l}}} \geq 0$

\subsubsection{The ``implicit'' search model}

The ``implicit'' search model is a McFadden Logit choice model modified to give null probability visits to doctors which afford patient $i$ non-positive utility, that way the free disposal requirement is fulfilled.

We call it the ``implicit'' search model because strictly speaking patient search is not formally included, yet one arrives at result qualitatively similar to specifications that do (like our own). The choice strategy noisily assigns positive probability to physicians who render $i$ high utility, and the higher this $u_{ij}$ expected utility is, the higher the probability of visit.

Instead of \textit{explicitly} defining search and subsequent choice, probability of visit depends upon expected utility from doctor $j$, $u_{ij}$, and a weighing parameter $\lambda$. The lower the value of $\lambda$, the noisier patient ``search'' is: they give high assingment probability to sub-optimal --yet feasible-- physician choices.

We define the components $s_{ij}$ of the patient's strategy vector $S_i$ as follows:

\begin{equation}
s_{ij} = \frac{\alpha_{ij}}{\sum_{k = 1}^{J} \alpha_{ik}}, \; \; \text{where } \; \alpha_{ij} = \begin{cases}
e^{\lambda u_{ij}}, \; \; \text{if } \; \; u_{ij} > 0 \\
0 , \; \; \text{if } \; \; u_{ij} = 0
\end{cases}
\label{eq:is_s}
\tag{S}
\end{equation}

Unlike the ``explicit'' search model, the probability that patient $i$ visits doctor $j$ is \textit{strictly} growing in $\kappa_i$, rather than being a piece-wise constant function with different levels. There is a discrete jump in probability at $\bar{\kappa_j}$, but elsewhere above $0$ the function is smoothly increasing, rewarding physicians with high $V_j$.

The same lemmas as in the non-search model still hold.


\begin{lemma}
\label{is_lemma}
No value in $(\frac{\tau}{V_j},\infty)$ can be the optimal solution of (\ref{eq:ns_k}).
\end{lemma}
    
    PROOF: APPENDIX
    
\begin{lemma}
\label{is_derivatives}
When $\bar{\kappa_j} \in [0,\frac{\tau}{V_j}]$,
\[
\frac{\partial Q_j}{\partial\bar{\kappa_j}} = \frac{\partial X_j}{\partial \bar{\kappa_j}}   
\]
\end{lemma}
    
PROOF: APPENDIX

Thus, the possible solutions for ($\ref{eq:ns_k}$) remain threefold: the corner solutions $0$ and $\infty$, and an inner solution where the simplified FOC $R^{\prime}(Q_j) = P^{\prime}(X_j)$ holds.

Equilibrium is achieved in a similar fashion, though now taking into account the patients' $S_i$ strategies.

\begin{equilibrium}[in ``implicit'' search]
    \label{is_eq}
Given a doctor-patient market specified by \\ $(\{(\kappa_i,\gamma_i)\}_{i =1}^{I},\{(V_j,R_j(\cdot),P_j(\cdot))\}_{i =1}^{J})$, we define an equilibrium as a set of physician thresholds $\{\bar{\kappa_j}\}_{j =1}^{J}$ satisfying (\ref{eq:ns_k}) and patient strategies $\{S_i\}_{i =1}^{I}$ as defined by (\ref{eq:is_s}), with their corresponding equilibrium (expected) patient demand and granted sick leave certificates $\{(Q_j,X_j)\}_{j =1}^{J}$ as described by (\ref{eq:s_Q}), (\ref{eq:s_X}), respectively.
\end{equilibrium}

\subsubsection{The ``explicit'' search model}

What we call the ``explicit'' model is in which patient strategy $S_i$ is explicitly the result of a sequential search algorithm on part of the patients. This is in line with \cite{schnell2017physician}, though made trickier by the fact that our model allows for physician services apart from contingent ``prescriptions'' (in our case, sick leave certificates). In our model some patients \textit{are} willing to visit physicians who don't intend to grant them sick leave, given a high enough benefit from the medical service proper, $V_j \kappa_i$.

FURTHER EXPLANATION REQUIRED, TO BE CONTINUED

\end{document}