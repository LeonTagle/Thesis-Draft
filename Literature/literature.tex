\documentclass[../main.tex]{subfiles}

\begin{document}

\section{Literature Review}


It is a well established factum that taking sick leave is subject to an economic calculus on part of the workers, rather than being an orthogonal, merely health-concerned matter. \cite{JohanssonPalme} begin their article with a quote by Nobel Laureate Ragnar Frisch: ``Regarding the high absence rate at the Department: Acquiring minor diseases, such as colds or flu, is an act of choice''. Their paper is among many others—\cite{italian-jobs}, \cite{norway-jobs}, \cite{us-jobs}, \cite{sweden-jobs}—which give empirical evidence of such a choice being driven by economic incentives, through an event study on exogenous institutional regime changes in the sampled nation's public insurance system. This line of research, however, is concerned with the actions of workers themselves and their subsequent effect on macroeconomic employment variables, whereas our main focus shall be the role played by physicians.

As for the literature on physicians themselves, it has become commonplace to regard them as agents with two sources of utility: their income and their ``altruistic'' interest on their patients' well-being. Empirical support for the latter can be found in experimental evidence both on medical students (\cite{avengers}, \cite{hs-wiesen}) as well as doctors themselves (\cite{hippocrates}, \cite{brosigkoch}). \cite{crea2019physician} finds no evidence for this, whereas \cite{godager2013profit} do, and explore its heterogeneity across physicians. The fact that physicians are also concerned with revenue, rather than being purely altruistic, is also well evidenced, see \cite{clemensgottlieb}, \cite{HSW}, \cite{autor}, and also \cite{rrk2012} for a review on the matter. Therein lies the dilemma with giving physicians the status of gatekeepers for different services and certifications, like disability insurance (as in autor). As \citet[p.~1]{markussen-roed} put it: ``In essence, the GPs [general practitioners] have been assigned the task of protecting the public (or private) insurer's purse against the customers who form the basis for their own livelihood''.

Both factors being well established in the literature on physicians, one strand of it would seek to design an optimal contract for medical care in the presence of such an economic calculus, see \cite{chone-ma} or \cite{optimal-altruism}; another, more in line with our approach, would evaluate the effects increased competition among doctors has on their rendered services. In general, \cite{currie2023effects} propose that increased competition would lead to physicians offering more services that please the clients yet relatively hurt their own utility (like drug prescriptions), and less services which bring them, physicians, more utility, at the expense of patient utility (like unwanted, expensive surgical procedures). \cite{iversen-luras} and \cite{iversen2004} provide empirical evidence that somewhat supports it: physicians with a shortage of customers will provide more services, thus obtaining more income \textit{per customer}. This line of work is more in line with our own approach, as the altruistic motivations of physicians are set momentarily aside and they're modeled as purely profit-seeking. This reasoning applies to scenarios like ours, where the issuance of sick leaves—particularly for short-term absences, as is the case in most situations\footnote{For cases where sickness leave is medically required we introduced the upper limit $\bar{\kappa}_{\max}$, such that physicians would never be so strict as to be negligent.}—does not have a significant impact on client health.

To our knowledge, the only article dealing specifically with sick leave certicate granting as a function of competition among physicians is \cite{markussen-roed}. \cite{cln} deal with sick leave as well, but make only the narrower point that in a Bayesian context doctors have almost no incentive to distrust patients' self-reported, unverifiable symptoms. \citeauthor{markussen-roed}'s methodology is similar to our own: after performing ``raw'' regression analysis, they set-up a model of patient choice as a McFadden Logit over observables $X_i$, including physician leniency (assumed to be observable for prospective patients), and as such can estimate the role leniency plays in demand for their services. They then perform a series of exercises, some of their findings include: physicians with variable wage (i.e. dependent on clientele) certify 7\% more absence days per month than fixed wage physicians; half of this difference has a causal interpretation, as observed from within-physician responses to market conditions; in general, more lenient gatekeeping gives the GP (physician) more customers, and more customers make the GP less lenient. Most of these stylized facts are taken up in our equilibrium models, in which patients take into account both sick leave ``leniency'' ($\bar{\kappa_j}$) as well as physician quality ($V_j$), such that physicians which offer better medical attention and thus enjoy higher demand can afford to be ``stricter'', and those who don't will have the incentive to gain clientele through leniency. Not captured is the physician fixed effect, the idiosyncratic motive to leniency: we assume leniency as merely a strategic choice based on expected demand, where prior to choosing leniency physicians only differ in ``quality'' ($V_j$) and visit revenue/cost of visit to the patient ($r_j, \tau_j$).

Despite the different subject matter, the main source of inspiration for this paper is \cite{schnell2017physician}, such that ours can be seen as an attempt to replicate her model and framework, intially devised for opioid markets, to the market for sick leaves. Schnell seeks to model the market scenario for the opioid crisis, with a primary market composed of physician prescriptions and a secondary black market. The presence of the latter, she concludes, makes unilateral interventions ineffective: curbing both excessive physician prescriptions as well as black market sales is required to make more than a dent on the number of opioids consumed in America.

Her paper includes four benchmark for the patient-physician market, building up to her main model including patient search, such that patients with a hire taste for opioids are assortatively matched with physicians more willing to prescribe them, and a secondary market. In our model we keep the former but not the latter, as the ``black market'' for sick leaves falls \textit{within} the primary market of physicians, composed of those willing to knowingly issue fraudulent sickness certificates. Search was repurposed to fit a two-dimensional frame of physicians, characterized both by their strictness as well as their service quality. What we call the ``explicit'' search model is in the vein of Schnell's sequential patient search, though more fleshed out in its dynamic programming framework for the reason just mentioned. Then there's the ``implicit'' model, which defines patient behavior according to a modified McFadden Logit. We show in the Appendix \ref{sec:schnell_logit} that such a framework wouldn't have altered Schnell's main conclusions.

Our model differs from Schnell's chiefly in the fact that physicians take into account \textit{other} physicians' behavior when selecting their own strategy, such that market equilibrium requires a Nash equilibrium in physician strategies. We discuss in the Appendix \ref{sec:separate} that the source of this feature, not present in Schnell's model, is the lack of additive separability across patients in the physician's utility function, such that her optimal behavior takes into account \textit{aggregate} patient demand as well as \textit{marginal}.

\end{document}